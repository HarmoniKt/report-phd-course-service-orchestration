% This is samplepaper.tex, a sample chapter demonstrating the
% LLNCS macro package for Springer Computer Science proceedings;
% Version 2.21 of 2022/01/12
%
\documentclass[runningheads]{llncs}
%
\usepackage[T1]{fontenc}
% T1 fonts will be used to generate the final print and online PDFs,
% so please use T1 fonts in your manuscript whenever possible.
% Other font encondings may result in incorrect characters.
%
\usepackage{graphicx}
% Used for displaying a sample figure. If possible, figure files should
% be included in EPS format.
%
% If you use the hyperref package, please uncomment the following two lines
% to display URLs in blue roman font according to Springer's eBook style:
%\usepackage{color}
%\renewcommand\UrlFont{\color{blue}\rmfamily}

\usepackage{xspace}
\newcommand{\thesystem}{\textit{HarmoniKt}\xspace}

\usepackage{acronym}
\acrodef{iiot}[IIoT]{Industrial Internet of Things}
\acrodef{mir}[MiR]{Mobile Industrial Robots}
\acrodef{spot}[SPOT]{Boston Dynamics Spot}

%
\begin{document}
%
\title{HarmoniKt: Microservices Middleware for\\Heterogeneous Robot Fleet Management\\
\small\textnormal{
    Project Report for the PhD Course \emph{Service Orchestration and Industrial IoT Platforms for Industry 4 and 5.0 environments} (A.Y. 2024/2025)
}
}
%
%\titlerunning{Abbreviated paper title}
% If the paper title is too long for the running head, you can set
% an abbreviated paper title here
%
\author{
Manuel Andruccioli\inst{1} \and
Angela Cortecchia\inst{1} \and
\\
Davide Domini\inst{1} \and
Nicolas Farabegoli\inst{1}
}
%
\authorrunning{F. Author et al.}
% First names are abbreviated in the running head.
% If there are more than two authors, 'et al.' is used.
%
\institute{
University of Bologna, Department of Computer Science and Engineering, Italy
\\
\email{\{manuel.andruccioli,angela.cortecchia,davide.domini,nicolas.farabegoli\}@unibo.it}
}
%
\maketitle              % typeset the header of the contribution
%
\begin{abstract}
This report presents a comprehensive analysis of \thesystem, a microservices-based middleware solution designed to provide unified access to heterogeneous robot fleets in \ac{iiot} contexts. The project successfully integrates \ac{spot} and \ac{mir} through a modular architecture that abstracts robot-specific functionalities behind standardized APIs. This analysis examines the project's architecture, implementation details, development evolution, and technical achievements.

\keywords{Microservices \and Middleware \and Robot Management.}
\end{abstract}
%
%
%
\section{First Section}





%
% ---- Bibliography ----
%
% BibTeX users should specify bibliography style 'splncs04'.
% References will then be sorted and formatted in the correct style.
%
% \bibliographystyle{splncs04}
% \bibliography{mybibliography}
%
\begin{thebibliography}{8}
\bibitem{ref_article1}
Author, F.: Article title. Journal \textbf{2}(5), 99--110 (2016)

\bibitem{ref_lncs1}
Author, F., Author, S.: Title of a proceedings paper. In: Editor,
F., Editor, S. (eds.) CONFERENCE 2016, LNCS, vol. 9999, pp. 1--13.
Springer, Heidelberg (2016). \doi{10.10007/1234567890}

\bibitem{ref_book1}
Author, F., Author, S., Author, T.: Book title. 2nd edn. Publisher,
Location (1999)

\bibitem{ref_proc1}
Author, A.-B.: Contribution title. In: 9th International Proceedings
on Proceedings, pp. 1--2. Publisher, Location (2010)

\bibitem{ref_url1}
LNCS Homepage, \url{http://www.springer.com/lncs}. Last accessed 4
Oct 2017
\end{thebibliography}
\end{document}
